\chapter{NFA $\to$ DFA}

\section{Objectives}

\begin{itemize}
	\item Convert an NFA to an NFA without $\emptystring$-transitions.
	\item Apply the subset construction algorithm to convert an NFA without $\emptystring$-transitions to an FA.
	\item [FIX?] Know the equivalent representations of Regular Languages.
\end{itemize}

\section{Introduction + Motivation}

\section{Removing $\emptystring$}

Suppose we have the following NFA:

\todo{Put in an NFA}

\subsection{$\emptystring$-Closure}

\begin{definition}
	Let $N = \nfatuple$ be an $\NFA$, and let $\arbitrarysubsetofstatesname \subseteq \setofstatesname$.
	Then the \emph{$\emptystring$-closure of $\arbitrarysubsetofstatesname$} is the smallest subset of states such that:
	\begin{enumerate}
		\item contains $\arbitrarysubsetofstatesname$;
		\item for any state $s \in \arbitrarysubsetofstatesname$, if $\transitionfunctionname(s, \emptystring) = s'$, then $s' \in \arbitrarysubsetofstatesname$.
	\end{enumerate}
\end{definition}

\section{The ``Powerset'' Construction (NFA $\to$ DFA)}

\begin{theorem}
	Every $\NFA$ has an equivalent $\DFA$ for the same language.
\end{theorem}

\begin{proof}
	Let $N = \nfatuple$ be an arbitrary $\NFA$.
	Then we create an equivalent $\DFA$ $D$ for the language of $N$ as follows\footnote{Both machines have the same alphabet.}:
	\begin{itemize}
		\item the states of $D$ correspond to subsets of states in $N$;
		\item the start state of $D$, $\startstatename'$, is the $\emptystring$-closure of $N$'s start state;
		\item the accept states of $D$ are the subsets of $N$'s states that have at least one accept state in $N$;
		\item the transition function $\transitionfunctionname'$ is defined as:
		\[
			\transitionfunctionname'(\arbitrarysubsetofstatesname, a) = E(\union_{x \in \arbitrarysubsetofstatesname} \transitionfunctionname(x, a)).
		\]
	\end{itemize}

We prove that the languages of $D$ and $N$ are the same, by showing that each language is a subset of the other.

$\Rightarrow$ ($L(D) \subseteq L(N)$): let $x \in L(D)$; we want to show $x \in L(N)$.
Because $x \in L(D)$, there is a computation of $x$ in $D$:
\[
	\startstatename' \xrightarrow{x_1} \cdots \xrightarrow{x_n} f
\]
where $f$ is an accept state of $D$.
Each of the states in this computation corresponds to a subset of $N$'s states; since $f$ is an accept state of $D$, this implies that the subset it corresponds to in $N$ contains an accept state. 
Since $N$ is an $\NFA$, this implies that there is an accepting computation of $x$ in $N$, implying $x \in L(N)$.

$\Leftarrow$ ($L(N) \subseteq L(D)$): let $x \in L(N)$; we want to show $x \in L(D)$.
Because $x \in L(N)$, there is a computation of $x$ in $N$:
\[
	\startstatename \xrightarrow{x_1} \cdots \xrightarrow{x_n} f,
\]
where $f$ is an accept state of $N$.
Consider the computation of $x$ in $D$; 
\todo{Prove via induction?}

\end{proof}

\section{Problems}