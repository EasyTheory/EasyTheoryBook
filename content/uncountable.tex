\chapter{Proving Uncountable}

\section{Objectives}

\begin{itemize}
	\item Outline the structure of a proof of uncountability.
	\item Prove that a set is uncountable.
\end{itemize}

\section{Introduction + Motivation}

\section{Steps to Prove Uncountable}

\begin{enumerate}
	\item[0.] State the claim.
	\item Assume the set is countable and give it a name.
	\item Observe that then there exists an enumeration of the elements of the set. Give a symbolic list of the elements.
	\item Show how the components and elements of the enumeration can be laid out in a 2D matrix. Each row corresponds to one element in the enumeration and each column corresponding to a component of the elements.
	\item Describe a particular element in the set that you show differs from each element at the component on the diagonal. 
	\item Show that the element described belongs to the set.
	\item Observe then that the described element must correspond to a specific element in the enumeration (and hence a row in the matrix).
	\item Show that the described element and the specific element (row of the matrix) cannot be the same.
\end{enumerate}

\section{Example}

\section{Problems}

\begin{enumerate}
	\item Prove that the following sets are uncountable.
	\begin{enumerate}
		\item The powerset of the natural numbers.
		\item The set of infinite sequences of 0s and 1s.
		\item The set of monotone-increasing functions, where such a function is of the form $f(n+1)>f(n)$ for all $n$.
	\end{enumerate}
	\item Determine which of the following are countable or uncountable:
	\begin{enumerate}
		\item The set of all non-deterministic Turing Machines.
		\item The set of all functions from $\{0, 1\}$ to $\natnums$.
		\item The set of all regular languages over $\{0, 1\}$.
		\item The set of all infinite subsets of $\natnums$.
	\end{enumerate}
\end{enumerate}