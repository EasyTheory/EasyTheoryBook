\chapter{Closure Operations for Regular Languages}

\section{Introduction + Motivation}

\section{Definitions}

\todo{Definition for closure in general under some operation}
Let $\mathcal{L}$ be a collection of languages, and $\oplus$ be some operation on languages. 
We say that \textit{$\mathcal{L}$ is closed under $\oplus$} if applying $\oplus$ to languages $L_1, \cdots L_k \in \mathcal{L}$ results in a language $L'$ that is also in $\mathcal{L}$.

For example, let $\mathcal{F}$ be the set of finite languages over the alphabet $\{0,1\}$. Then $\mathcal{F}$ is closed under union of two languages because the union of any two finite languages is also finite.
They are also closed under intersection for analogous reasons.

However, $\mathcal{F}$ is \textit{not} closed under complement, because the complement of a finite language over the alphabet $\{0,1\}$ is not finite. 
For example, the complement of the finite language $\{0, 10\}$ is the set $\{\Epsilon, 1, 00, 01, 11, 000, \cdots\}$, which has infinitely many strings.

\section{Closure Under Complement}

\begin{theorem}
	Regular languages are closed under complement.
\end{theorem}

\begin{proof}
	It suffices to show that for each regular language $L$, there is a \DFA for $\complement{L}$.
	Let $M$ be any \DFA for $L$. For any string $w$ that $M$ accepts, the computation of $M$ on $w$ ends in some accept state; and for any string $x$ that $M$ does \textit{not} accept, that computation ends in some non-accept state.
	
	We can create a \DFA $\complement{M}$ for $\complement{L}$ by swapping the accept and non-accept states of $M$.
	\todo{Figure demonstrating the conversion.}
	Therefore, any computation that ends in an accept state of $\complement{M}$ ended in a non-accept state of $M$, and vice versa. 
\end{proof}

\section{Closure Under Union and Intersection}

\begin{theorem}
	Regular languages are closed under union.
\end{theorem}

\todo{Figure for the product construction}

\begin{corollary}
	Regular languages are closed under intersection.
\end{corollary}

\begin{proof}
	For any two languages $L_1, L_2$, it is the case that
	\[
		L_1 \cap L_2 = \complement{\complement{L_1} \union \complement{L_2}},
	\]
	known as DeMorgan's Law. 
	Since we have shown that regular languages are closed under union and complement, they must be closed under intersection as well.
\end{proof}

\section{Problems}

\todo{Symmetric difference}

\todo{Number of final states for union/intersection}