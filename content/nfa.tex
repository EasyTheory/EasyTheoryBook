\chapter{Nondeterministic Finite Automata}

\section{Objectives}

\begin{itemize}
	\item Define nondeterministic finite automata (NFA).
	\item Define the $\emptystring$-closure of a set of states.
	\item Design NFA.
	\item Compute the $\emptystring$-closure of a set of states.
\end{itemize}

\section{Introduction + Motivation}

\section{Definitions}

A \textit{nondeterministic finite automaton} (\NFA) consists of 5 parts:
\begin{itemize}
	\item $\setofstatesname$, a finite set of \textit{states};
	\item $\startstatename$, a state in $\setofstatesname$ called the \textit{start state};
	\item $\acceptstatesname$, a subset of $\setofstatesname$ called the \textit{accept states};
	\item $\inputalphabetname$, a finite set, called the \textit{input alphabet};
	\item $\transitionfunctionname$, a function from $\setofstatesname \times \left(\inputalphabetname \cup \{\emptystring\}\right)$ to $\powerset{\setofstatesname}$, called the \textit{transition function}.
\end{itemize}

Notice that the only real difference between an $\NFA$ and a $\DFA$ is the transition function, where:
\begin{enumerate}
	\item the input pair is $\setofstatesname \times \left(\inputalphabetname \cup \{\emptystring\}\right)$ for an $\NFA$, and $\setofstatesname \times \inputalphabetname$ for a $\DFA$.
	\item the output is $\powerset{\setofstatesname}$ for an $\NFA$, and just $\setofstatesname$ for a $\DFA$.
\end{enumerate}

\section{Examples}

\todo{All strings that contain 10 as a substring}

\section{Closure Under Union, Concatenation, and Star}

\todo{Union/concat/star boxes}

\section{Problems}

\todo{Symmetric difference}

\todo{Number of final states for union/intersection}