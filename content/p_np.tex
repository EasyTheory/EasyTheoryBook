\chapter{P, NP, NP-Completeness, and Boolean Satisfiability}

\section{Objectives}

\begin{itemize}
	\item Describe where we are in classifying problems.
	\item Describe what it means for a problem to be NP-Complete.
	\item Be able to determine the satisfiability of a Boolean formula.
\end{itemize}

\section{Introduction + Motivation}

For the remainder of the course, let's focus only on problems $A$ that \emph{are} solvable.
Clearly, they must take some amount of time/resources/etc. to solve; the question we want to address is: how \emph{much} such resources does an algorithm need to solve some problem $A$?

\section{P and NP}

Until recently,\footnote{This has changed to having ``fast'' representing ``log-linear'' runtime (i.e., of the form $O(n \log^c n)$ for some constant $c$), and any runtime ``more'' than this is ``slow.''} one important measure of distinguishing ``slow'' from ``fast'' algorithms was that when they ran in ``polynomial time,'' then they were ``fast''; otherwise, they were ``slow.''

\section{NP-Complete Problems}

\subsection{How to Prove a Problem is NP-Complete}

The steps to proving a problem $A$ is $\NP$-complete are:
\begin{enumerate}
	\item \textbf{Prove $A \in \NP$}. Describe how a ``guessed'' solution can be verified in deterministic polynomial time.
	\item \textbf{Prove that $A$ is $\NP$-hard}. Reduce a known $\NP$-hard problem to $A$. Follow the following steps to accomplish this:
	\begin{enumerate}
		\item State the known $\NP$-hard problem $B$.
		\item Explain the abstract view of the reduction.
		\todo{Add figure of reduction}
		\item Describe the transformation.
		\item Prove the transformation can be accomplished in deterministic polynomial time.
		\item \underline{Prove a solution to $A$ is equivalent to a solution to the corresponding instance of $B$.} (This is the most important step.)
	\end{enumerate}
	\item \textbf{Summarize and Conclude}.
\end{enumerate}

\section{Boolean Satisfiability}

The ``first'' $\NP$-complete problem is \emph{boolean satisfiability}.

\problem{Satisfiability (SAT)}{
	A set $U$ of variables and a collection $C$ of clauses over $U$.
}{
	Is there a truth assignment to the variables in $U$ that satisfies $C$?
}

\section{Problems}