\chapter{DFA Minimization}

\section{Objectives}

\begin{itemize}
	\item Apply the minimization algorithm to finite automata.
\end{itemize}

\section{Introduction + Motivation}

We want to minimize the size of a finite automaton as much as possible, \emph{as long as it has the same language as that of the original.}

\section{Minimization}

Think about the following two states:

\todo{fix the transition labels and state names}
\begin{tikzpicture}[scale=0.7, shorten >=1pt,node distance=5cm,on grid,auto,initial text={},every state/.style={align=center},line width=0.4mm] 
	\node[state] (q_0)   {$q_0$}; 
	\node[state] (q_1) [below=of q_0] {$q_1$}; 
	\node[state] (q_2) [right=of q_0] {$q_2$}; 
	\node[state] (q_3) [below=of q_2] {$q_3$}; 
	\path[->] 
	(q_0) edge node {$a$} (q_2)
	(q_0) edge node {$b$} (q_3)
	(q_1) edge node {$a$} (q_2)
	(q_1) edge node {$b$} (q_3)
	;
\end{tikzpicture}

Clearly these two states ``do the same thing,'' since their transitions go to identically the same states, given the same input from both.
Here, we can just merge the two states together:

\todo{Same example, one state on the left with the name containing the original two}

However, it may be more complicated to  figure out, say, this example of what states are the ``same'':

\todo{Bigger example}

One attempt to minimize an automaton $M$ is to try all ``smaller'' automata $m$, and check if the languages of $M$ and $m$ are the same.
Several problems arise: if $M$ is moderately big, the number of automata smaller than it is \emph{enormous}.
Setting that aside, how would one check  if $M$ and $m$ have the same language, in general? Note that $M, m$ have to accept the same strings, \emph{and not accept} the same strings.
If you have a small example of $M, m$ in front of you, maybe that can be done by inferring some properties of both. 
But in general, how would that be done? Checking all infinitely many strings against both machines clearly will take too long, so we need a better idea.

\subsection{Distinguishing States}

Instead of thinking about two states $p, q$ and reading a single character, let's think about \emph{every possible string that can be read from $p, q$}.
If, for \emph{any} string $w \in \Sigma^\star$, we have that reading $w$ from $p, q$ has the resulting states $p', q'$ either both accepting or both rejecting, then $p, q$ are ``equivalent'' in terms of acceptance. 
Note that the resulting states \emph{may not be identically the same state}.

This generalizes our idea before: the two states above went to the same states on the same transitions, so anything read after that must be identically the same state from both, hence they must agree.
However, this might not always be the case, as in this example:

\todo{Bigger example again}

But now suppose we have this example, where $p', q'$ are states we \emph{already know} are not equivalent:

\todo{$p, q, p', q'$ example}

Then we must know that $p, q$ are not equivalent either, because if they were, then $p', q'$ must be, since reading any string from $p, q$ results in the same state.
This yields a simple algorithm:

%\begin{algorithm}
%	\caption{MinimizeFA}
%	\begin{algorithmic}[1]
%		\State Input: $\FA$ $M$
%		\State Output: $\FA$ $M'$ with the same language as $M$, and minimum number of states possible.
%		\EndProcedure
%	\end{algorithmic}
%\end{algorithm}

\section{Example}

\section{Problems}