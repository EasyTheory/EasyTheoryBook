\chapter{Introduction and Review: Logic, Proofs, Sets, and Functions, Relations, and Regular Expressions}

\section{Objectives}

\begin{itemize}
	\item Describe and apply basic operations of logic.
	\item Describe and apply basic operations of set theory (and languages as sets of strings).
	\item Define the properties of functions and equivalence relations.
	\item Review and practice reading and writing regular expressions.
\end{itemize}

\section{Introduction + Motivation}

\section{Definitions}

Some common sets are the \emph{natural numbers} $\natnums$, which are the numbers $0, 1, 2, 3, \cdots$; the \emph{integers} $\integers$, which are the numbers $\cdots, -3, -2, -1, 0, 1, 2, 3, \cdots$; and the \emph{real numbers} $\reals$.

A \emph{set} is a collection of ``elements'' without duplicates, usually written with curly braces and not in any particular order, like: \{3, 4, 2, ``New York''\}.
If $A$ and $B$ are two sets, then the notation $A \subseteq B$ indicates that $A$ is a \emph{subset} of $B$, in that every element of the set $A$ is also in the set $B$. An example is $A = \{0, 1, 4\}$ and $B = \{0, 1, 4, 7\}$.
If $A \subseteq B$ and some element of $B$ is also not in $A$, $A$ is a \emph{proper} subset of $B$, written $A \subset B$ (and the previous example shows that $A$ is a proper subset of $B$).
The set with no elements is called the \emph{empty set}, written $\emptyset$.

If $A$ is a set, then $\powerset{A}$ is the \emph{power set} of $A$, which is the collection of all subsets of $A$, including $A$ itself.

\section{Examples}

\section{Problems}

\begin{enumerate}
	\item Let $L = \{a, b, c\}$; compute the set $LL$.
	\item Let $L = \{0, 1, 2, 3, 4, 5, 6, 7, 8, 9\}$; describe the language $L^+$ in English.
	\item For each of the functions below, indicate which of the properties hold: total, one-to-one, onto, and/or bijection.
	\begin{enumerate}
		\item $m: \natnums \to \natnums$ defined by $m(x) = \min(x, 2)$.
		\item $s: \natnums \to \reals$ defined by $s(x) = \sqrt{x}$.
		\item $d: \natnums \to \reals$ defined by $d(x) = \frac{1}{x}$.
		\item $i: \natnums \to \natnums$ defined by $i(x) = x$.
		\item $c: \reals^{0+} \to \natnums$ defined by $c(x) = \lceil x \rceil$, where $\reals^{0+}$ indicates non-negative real numbers.
		\item $e: \natnums \to \natnums$ defined by $e(x) = 2x$.
		\item $f: \natnums \to \integers$ defined by:
		\[
			f(x) = 
			\begin{cases}
				\frac{x}{2},& \text{if $x$ is even}\\
				\frac{-x-1}{2},& \text{otherwise}
			\end{cases}
		\]
	\end{enumerate}
	
	\item For the following languages, show a corresponding regular expression.
	\begin{enumerate}
		\item All strings in $\{0,1\}^\star$ containing exactly two 0's.
		\item All strings in $\{0,1\}^\star$ containing at least two 0's.
		\item All strings in $\{0,1\}^\star$ that do not end in 01.
	\end{enumerate}

	\item  In each case below, find a string of minimum length in $\{a, b\}^\star$ not in the language corresponding to the given regular expression.
	\begin{enumerate}
		\item $b^\star (ab)^\star a^\star$
		\item $(a^\star \orop b^\star)(a^\star \orop b^\star)(a^\star \orop b^\star)$
		\item $a^\star (baa^\star)^\star b^\star$
		\item $b^\star (a \orop ba)^\star b^\star$
	\end{enumerate}
\end{enumerate}