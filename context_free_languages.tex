\part{Context-Free Languages}

\section{CFGs}

\begin{enumerate}[resume]

\item \label{summer_2017_cfg} \Level{1} Produce a CFG $G$ for $L_1 \cup L_2$ where $L_1 = \{0^n 1^n 2^m 3^m : n,m \ge 0 \}$ and $L_2 = \{0^n 1^m 2^m 3^n : n, m \ge 0 \}$.

\item \Level{1} Recall the definition of $Double(L)$ from \cref{ex2017fsquaredouble}. Show that if $L$ is a CFL, then $Double(L)$ is also a CFL.

\item \Level{2} Give an example of a context-free language $L$ that, for \emph{any} CFG with language $L$, requires \emph{at least} 2 variables. 
Prove that there is no CFG for $L$ with only 1 variable.\footnote{Unfortunately the problem of determining if a given CFG is minimal with respect to the number of variables is undecidable, so brute-force may or may not work here.}

\item \Level{4} Consider the string $a^n$ for some integer $n$. The language $\{a^n\}$ clearly is regular (it has only 1 string), and hence is context-free. A simple CFG for it is $S \to a\cdots a$. 

We will consider the \emph{complexity of $a^n$} to be the following: we sum for every rule the number of variables and terminals in that rule as well as the $\to$ symbol. For example, the complexity of $a^n$ is at most $n+2$ since the grammar above has 1 variable, $n$ occurrences of the symbol $a$, and one $\to$ symbol.

We can improve this in some cases. For $a^9$, we can make the grammar $S \to BBB, B \to aaa$, which has 5 variable occurrences, 2 $\to$ occurrences, and 3 terminals, which sums to 10 (better than the trivial bound of 11).

For what $n$ can we have a better complexity than $n+2$, and what $n$ require $n+2$?
Write a program to generate the complexity (with a CFG with that complexity) for all $n \le 100$. 
Discuss how you designed your program, as well as discuss what you believe the ``correct'' answer is. 
Can you say much about complexities for much larger $n$ (say, at least $10^5$)?

\end{enumerate}

\section{CNF}

\begin{enumerate}[resume]

\item \Level{2} Convert the CFG in \cref{ex2017fcfg} into CNF.

\item \Level{2} Consider any regular grammar. If we wanted to convert it into CNF, show that one of the five steps in the conversion process can be skipped.

\end{enumerate}

\section{PDAs}

\begin{enumerate}[resume]

\item \label{ex2017fcfg} \Level{1}  Convert the following CFG into an equivalent PDA.
\begin{align*}
S &\to aBb\\
B &\to ac \;|\;S\;|\;\varepsilon
\end{align*}


\item \Level{1} Produce an equivalent PDA $P$ for the grammar in \cref{summer_2017_cfg}.

\item \Level{2} Prove or disprove the following statements:
\begin{enumerate}
	\item Stacks are getting expensive lately! I limit my PDAs to only have a maximum stack height of 50, and if during a computation the PDA tries to push on a stack with height of 50, the computation terminates. My model of PDAs then still recognizes the context-free languages.
	\item The energy it is taking for PDAs to take a transition where it reads off of the input is increasing, so it may be a good idea to allow ``breaks" between them. I want every accepting computation of every string to consist of at least 1 $\varepsilon$-transition between each character read from the input, without worrying about what happens to the stack. Call these machines $\varepsilon$-PDAs. Then $\varepsilon$-PDAs still recognize the context-free languages.
\end{enumerate}

\item \Level{3} Consider two strings $w = w_1 \cdots w_n$ and $x = x_1 \cdots x_m$. We want to consider \emph{edits}: insertions, deletions, and substitutions. All such edits are equal; they have a ``cost'' of 1.

The \emph{edit distance of $w, x$}, denoted ${\textsf{ED}}(w,x)$, is the minimum size of a sequence of edits over all possible sequences of edits on $w$ to have $w$ equal to $x$. For example, if $w = abb$ and $x = bc$, then the (minimum) edit distance is 2; one such edit sequence is deleting $a$, and substituting the second $b$ for a $c$. Also, there is no edit sequence of length 1 to transform $abb$ into $bc$.

Consider the following language for a given value of $k$ (where $x^{rev}$ is the reverse of string $x$):
\[
\textsf{Edit(k)} = \{ w \# x^{rev} : w, x \in \{a, b, c\}^\star, \textsf{ED}(w,x) \le k  \}.
\]
In other words, this language has all strings of the form $w\# x^{rev}$ where the edit distance is less than $k$.
\begin{enumerate}
	\item Produce a PDA $P$ that recognizes $\textsf{Edit(1)}$.
	\item Convert $P$ into an equivalent CFG $G$ using the techniques from class (Hint: write a program to output all the rules.)
	\item Show that for any fixed value of $k$ that $\textsf{Edit}(k)$ is context-free. (Hint: generalize $\textsf{Edit}(1)$.)
\end{enumerate}

\item \Level{3} A \emph{recursive automaton} is a 3-tuple $(M, P, \{D_k : k \in M \})$, where $M$ is a finite set of \emph{module names}, $P \in M$ is the \emph{start module}, and for each $k \in M$, there is an NFA $D_k$ over the alphabet $\Sigma \cup M$, and have pairwise disjoint states. 

In other words, the recursive automaton is a collection of NFAs that can ``call'' other NFAs in $M$ via a transition, including themselves. 
The notion of computation from NFAs extends to here as well, except now we keep a \emph{stack} of which modules ``called'' which other modules, and in what order. 
We push onto this stack if we call another module, and we pop if we end in a final state of that module (and return to the module then on the top of the stack).
We accept a string iff the string is in a final state of $P$ (the start module) and the stack of module calls is empty.
		
Prove that the languages of recursive automata are exactly the CFLs.
\end{enumerate}

\section{Pumping Lemma for Context-Free Languages}

\begin{enumerate}[resume]
\item \Level{2} For a language $L$, let $\textsf{Shuffle}(L)$ be the following language:
\[
\{ w : x \in L, |w| = |x|,\;\text{and $w$ is a permutation of the letters in $x$} \}.
\]
\begin{enumerate}
	\item Prove that regular languages are \emph{not} closed under shuffle. (Hint: use part (b))
	\item Prove that context-free languages are \emph{not} closed under shuffle.
\end{enumerate}

\item \Level{2} Let $L = \{w \in \{0, 1\}^\star : |w|\;\text{is a power of $n$, and $n$ is even}\}$. Prove that $L$ is not context-free.

\item \Level{2} Recall the definition of $Square(L)$ from \cref{ex2017fsquaredouble}. 
	Show that there is a CFL $L$ for which $Square(L)$ is not a CFL.

\item \Level{2} Produce two non-context-free languages $L_1, L_2$ such that $L_1 \cap L_2$ is context-free.

\item \Level{3} Produce a non-context-free language $L$ such that $L^\star$ is context-free.

\item \Level{4} Produce a non-context-free language $L$ such that $\overline{L}$ is also non-context-free.

\item \Level{4} Produce two non-context-free languages $L_1, L_2$ such that $L_1 L_2$ is context-free.

\item \Level{4} Show that any unary context-free language is regular.
\end{enumerate}

